\documentclass[a5paper,10pt]{article}
\usepackage[portuguese]{babel}
%\usepackage[latin1]{inputenc}
\usepackage[utf8]{inputenc}
\usepackage[T1]{fontenc} 
\usepackage{graphicx}
\usepackage[pdftex,plainpages=false,pdfpagelabels,
pagebackref=false,colorlinks=true,citecolor=blue,
linkcolor=black,urlcolor=blue,filecolor=blue,
bookmarksopen=true]{hyperref}
\usepackage{amsmath,amsfonts,mathrsfs,amssymb}
\usepackage{multicol,parskip}
\usepackage{fancyhdr}
\usepackage{indentfirst}
\usepackage{makeidx} 
\usepackage[papersize={5.25in,8.26in},left=1cm,
right=1cm,top=2cm,bottom=2cm]{geometry}
\usepackage{fourier-orns}
\usepackage{mdframed,xcolor}
\usepackage{sectsty}
\usepackage{enumitem}

%%%%%%%%%%%%%%%%%%%%%%%%%%%%%%%%%%%%%%%%%%%%%%%%%%%%%%%%%%%%%%%%%%%%%%
\def\ni{\noindent}
\def\be{\begin{equation}}
\def\ee{\end{equation}}

%%%%%%%%%%%%%%%%%%%%% Box Reflita %%%%%%%%%%%%%%%%%%%%%%%%%%%%%%%%%%
\newmdenv[shadow=false, shadowsize=2pt, linewidth=2pt,
frametitlerule=true, roundcorner=20pt,]{reflita}


%%%%%%%%%%%%%%%%%%%%%%%%%%%%%%%%%%%%%%%%%%%%%%%%%%%%%%%%%%%%%%%%%%%%%%
\numberwithin{equation}{section}
\setlength{\parskip}{2pt}
\setlength{\parindent}{18pt}
\sectionfont{\large}

\pagestyle{fancy}
\renewcommand{\footrulewidth}{0.5pt}
\setlength{\headheight}{13.6pt}
\renewcommand{\baselinestretch}{1.0}
\fancyhead{}
\lhead{}
\rhead{\nouppercase\leftmark}


\begin{document}
\pagestyle{empty}


%%%%%%%%%%%%%%%%%%%%%%%%%%%%%%%%%%%%%%%%%%%%%%%%%%%%%%%%%%%%%%%%%%%%%%
\section*{Prefácio}
%\addcontentsline{toc}{section}{Prefácio}

\pagestyle{empty}
\newpage
\tableofcontents
\newpage
\listoffigures
\newpage
\pagenumbering{arabic}

\setcounter{page}{1}
\pagestyle{fancy}
\section{Introdução}

\section{Cálculo Diferencial}


\section{Álgebra Linear}

\begin{appendix}

%%%%%%%%%%%%%%%%%%%%%%%%%%%%%%%%%%%%%%%%%%%%%%%%%%%%%%%%%%%%%%%%%%%%%
\section{Apêndice I}

%%%%%%%%%%%%%%%%%%%%%%%%%%%%%%%%%%%%%%%%%%%%%%%%%%%%%%%%%%%%%%%%%%%%%%
\subsection{Open Course Ware e MOOC's}

Nesta seção garimpamos alguns materiais disponibilizados gratuitamente na 
internet. A maioria dos materiais estão em inglês, mas são de plataformas 
reconhecidas e confiáveis, de empreendimentos como os do MIT OCW, EdX, 
Coursera, etc. Abaixo links de alguns open course famosos.

\href{https://ocw.mit.edu/courses/find-by-department/}{MIT Open
Courseware (MIT OCW)}: Iniciativa do MIT em prover educação
online de qualidade para o mundo. O site mostra os cursos separados 
por departamentos.

\href{https://lagunita.stanford.edu/}{Stanford Online Lagunita}: 
Open Course Ware de Stanford, contém dezenas de cursos completos,
online e gratuitos.

\href{https://online-learning.harvard.edu/}{Harvard Online 
Learning}: Cursos Online Gratuitos da Universidade Harvard.

\href{https://www.youtube.com/stanford}{Canal da Stanford University
no Youtube}: Contém diversos cursos completos, todos gravados em
formato de aula online.

\href{https://www.coursera.org/}{Coursera}: É um dos MOOC mais
difundidos atualmente. Apresenta uma grande variedade de cursos
provindos de grandes universidades como Stanford, Harvard, MIT,
etc.

\href{https://www.edx.org/}{EdX}: É um dos MOOC mais
difundidos atualmente. Apresenta uma grande variedade de cursos
provindos de grandes universidades como Stanford, Harvard, MIT,
etc.


\end{appendix}

\printindex
\end{document}
